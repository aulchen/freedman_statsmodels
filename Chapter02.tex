\documentclass{article}
\usepackage{amsmath}
\usepackage{amsfonts}
\usepackage{amssymb}
\usepackage{bbm}
\usepackage{mathabx}

\newenvironment{proof}{\paragraph{Proof:}}{\hfill$\square$}
\newtheorem{theorem}{Theorem}
\newtheorem{lemma}[theorem]{Lemma}
\newtheorem{corollary}[theorem]{Corollary}

\usepackage{graphicx}
\graphicspath{ {./images/Chapter03/} }

\newcommand{\R}{\mathbb{R}}
\newcommand{\Q}{\mathbb{Q}}
\newcommand{\Z}{\mathbb{Z}}
\newcommand{\N}{\mathbb{N}}

\newcommand{\F}{\mathcal{F}}

\newcommand{\prob}{\boldsymbol{P}}
\newcommand{\E}{\text{E}}
\newcommand{\var}{\text{Var}}
\newcommand{\cov}{\text{Cov}}
\newcommand{\sd}{\text{SD}}
\newcommand{\se}{\text{SE}}

\newcommand{\pois}{\text{Pois}}

\author{Arthur Chen}
\title{Chapter 2 The Regression Line}
\date{\today}

\begin{document}

\maketitle

\section*{Exercise Set A}

\subsection*{Problem 1}

In the Pearson-Lee data, the average height of the fathers was 67.7 inches; the SD was 2.74 inches. The average height of the sons was 68.7 inches; the SD was 2.81 inches. The correlation was .501.

\subsubsection*{Part a}

True or false and explain: because the sons average an inch taller than the fathers, if the father is 72 inches tall, it's 50-50 whether the son is taller than 73 inches.

False. The statement as given states that given the point $(\bar{x}, \bar{y})$, moving to $\bar{x} + c$ causes the conditional mean of $y$ to increase to $\bar{y} + c$. This is incorrect. The correct statement is that by moving to $\bar{x} + SD_x$, the conditional mean of $y$ increases to $\bar{y} + r SD_y$.

\subsubsection*{Part b}

Find the regression line of son's height on father's height, and its RMS error.

The line is

\[
y - 68.7 = .501 \frac{2.81}{2.74}(x - 67.7)
\]

From the formulas, the MSE is

\[
MSE = (1-.501^2)(2.81^2) = 5.914
\]

and RMS is $\sqrt{5.914} = 2.432$.

\subsection*{Problem 2}

Can you determine $a$ in the regression equation by measuring the length of the spring with no load? With one measurement? Ten measurements? Explain.

No. We never can determine $a$. We can only estimate it from the observed $(x, y)$ values.

\section*{Exercise Set B}

\subsection*{Problem 9}

Equation 7, reprinted below, is a model.

\[
Y_i = a + bx_i + e_i
\]

\subsection*{Problem 10}

In the equation from Problem 9, $a$ and $b$ are unobservable parameters. $Y_i$ is an observable random variable, while $e_i$ is an unobservable random variable.

\subsection*{Problem 11}

According to Equation 7, the 439.00 in table 1 is the observed value of a random variable.

\subsection*{Problem 13}

A statistician has a sample, and is computing the sum of the squared deviations of the sample numbers from a number $q$. The sum of the squared deviations will be smallest when $q$ is the average of the measurements in the sample. It's well known that letting your estimate of a sample be the mean minimizes the mean-squared error between the estimate and the sampled points.

\end{document}