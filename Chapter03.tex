\documentclass{article}
\usepackage{amsmath}
\usepackage{amsfonts}
\usepackage{amssymb}
\usepackage{bbm}
\usepackage{mathabx}

\newenvironment{proof}{\paragraph{Proof:}}{\hfill$\square$}
\newtheorem{theorem}{Theorem}
\newtheorem{lemma}[theorem]{Lemma}
\newtheorem{corollary}[theorem]{Corollary}

\usepackage{graphicx}
\graphicspath{ {./images/Chapter03/} }

\newcommand{\R}{\mathbb{R}}
\newcommand{\Q}{\mathbb{Q}}
\newcommand{\Z}{\mathbb{Z}}
\newcommand{\N}{\mathbb{N}}

\newcommand{\F}{\mathcal{F}}

\newcommand{\prob}{\boldsymbol{P}}
\newcommand{\E}{\text{E}}
\newcommand{\var}{\text{Var}}
\newcommand{\cov}{\text{Cov}}
\newcommand{\sd}{\text{SD}}
\newcommand{\se}{\text{SE}}

\newcommand{\pois}{\text{Pois}}

\author{Arthur Chen}
\title{Chapter 3 - Matrix Algebra}
\date{\today}

\begin{document}

\maketitle

\section*{Problem Set C}

\subsection*{Problem 3}

If $A$ is a fixed (i.e. nonrandom) matrix of size $n \times \ell$ and $B$ is a fixed matrix of size $1 \times m$, show that $\E(AUB) = A\E(U)B$.

$AU \in \R^{n \times 1}$. The $i$th element is

\[
(AU)_i = \sum_{k=1}^\ell a_{ik}u_k
\]

and the $i,j$th element of $AUB \in \R^{n \times m}$ is

\[
(AUB)_{ij} = \sum_{k=1}^\ell a_{ik}u_k b_j
\]

The expectation of a random matrix is the expectation of each of its elements, so

\[
\E(AUB)_{ij} = \sum_{k=1}^\ell a_{ik} \E(u_k) b_j
\]

Unwinding the matrix multiplications shows that this equals $A\E(U)B$, as desired.

This result generalizes to arbitrary matrices. Let $A \in \R^{n \times d}, U \in \R^{d \times c}$, and $B \in \R^{c \times m}$. We first calculate the elements of $A\E(U)B$. Given

\[
(UB)_{ij} = \sum_{\ell = 1}^c U_{i \ell}B_{ell j}
\]

\begin{align*}
(AUB)_{ij} &= \sum_{k=1}^d A_{ik}(UB)_{kj} \\
&= \sum_{\ell = 1}^c \sum_{k=1}^d A_{ik} U_{k \ell}B_{\ell j}
\end{align*}

The expectation of a random matrix is the elementwise expectation of each of its elements, so

\[
(E(AUB))_{ij} = E((AUB)_{ij}) = \sum_{\ell = 1}^c \sum_{k=1}^d A_{ik} \E(U)_{k \ell}B_{\ell j}
\]

and by unwinding this, we see that this equals $A\E(U)B$, as desired.

\subsection*{Problem 4}

Show that $\cov(AU) = A\cov(U)A'$.

By definition, $\cov(AU) = \E\left((AU-\E(AU))(AU-\E(AU))'\right)$. Looking at the $AU-\E(AU)$ term, by Problem 3,

\[
AU-\E(AU) = AU - A\E(U) = A(U-E(U))
\]

Similarly,

\[
(AU-\E(AU))' = (U-\E(U))'A'
\]

and so

\begin{align*}
\cov(AU) &= \E\left((AU-\E(AU))(AU-\E(AU))'\right)  \\
&= \E\left( A(U - \E(U))(U - \E(U))' A' \right) \\
&= A \E\left((U - \E(U))(U - \E(U))'  \right) A'
= A \cov(U) A'
\end{align*}

where the last result follows from Problem 3 because $A$ and $A'$ are fixed, while $(U - \E(U))(U - \E(U))'$ is random.

\end{document}